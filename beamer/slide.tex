%\documentclass[draft]{beamer} %temporary render
\documentclass{beamer} %final render

\usepackage[french]{babel} 	%utilisation des
\usepackage[T1]{fontenc} 	%caracteres
\usepackage[utf8]{inputenc} 	%francais
\usepackage{graphicx}
\usepackage{color}
\usepackage{ccaption}
\usepackage{fancyvrb}
\usepackage{verbatim}
\usepackage{float}
\usepackage{csquotes}	%pour les citations
\usetheme{Frankfurt}
\usepackage{marvosym} % \MVRIGHTarrow

\AtBeginSection[] 	%va permettre fl'affichage du sommaire avant
{					%chaque vouvelle partie
  \begin{frame}
    \frametitle{Plan}
    \tableofcontents[currentsection, hideothersubsections]
  \end{frame} 
}

\setbeamertemplate{navigation symbols}{}
\setbeamertemplate{footline}[frame number]
\setbeamercolor{footline}{fg=gray,bg=black}

\makeatletter
\newcommand\titlegraphicii[1]{\def\inserttitlegraphicii{#1}}
\titlegraphicii{}
\setbeamertemplate{title page}
{
  \vbox{}
   {\usebeamercolor[fg]{titlegraphic}\inserttitlegraphic\hfill\inserttitlegraphicii\par}
  \begin{centering}
    \begin{beamercolorbox}[sep=8pt,center]{institute}
      \insertinstitute
    \end{beamercolorbox}
    \begin{beamercolorbox}[sep=8pt,center]{title}
      \usebeamerfont{title}\inserttitle\par%
      \ifx\insertsubtitle\@empty%
      \else%
        \vskip0.25em%
        {\usebeamerfont{subtitle}\usebeamercolor[fg]{subtitle}\insertsubtitle\par}%
      \fi%     
    \end{beamercolorbox}%
    \vskip1em\par
    \begin{beamercolorbox}[sep=8pt,center]{author}
      \usebeamerfont{author}\insertauthor
    \end{beamercolorbox}
    \begin{beamercolorbox}[sep=8pt,center]{date}
      \usebeamerfont{date}\insertdate
    \end{beamercolorbox}%\vskip0.5em
  \end{centering}
  %\vfill
}
\makeatother
\author{Benjamin BARBESANGE et Benoît GARCON}
\title{Soutenance de Projet Ingénieur 3ème Année}
\subtitle{WatchDogZZ - Suivi de personnes dans les bâtiments}
%\institute{Siemens Industry Software\\ISIMA}
\date{Projet de 120h}
% \titlegraphicii{\includegraphics[height=5mm]{watchdogzz.png}}
\titlegraphic{\includegraphics[height=5mm]{isima.png}}

\logo{\includegraphics[height=10mm]{logo.png}} %le logo en bas a droite

\begin{document}

	\begin{frame} %frame de titre
		\maketitle
    \vspace*{1cm}
    \footnotesize
    \begin{tabular}{ll}
      Tuteur projet &: Pierre COLOMB\\
      Référent ISIMA &: Eva HASSINGER
    \end{tabular}
	\end{frame}

%---------- Introduction

  \section{Introduction}
  \begin{frame}{\secname}
    
  \end{frame}

%---------- Plan

  \begin{frame}{Plan}
    \tableofcontents
  \end{frame}

%---------- Etude

  \section{Solution}
  \begin{frame}

  \end{frame}

  \subsection{Android}
  \begin{frame}{\subsecname}  

  \end{frame}

  \subsection{Service Web}
  \begin{frame}{\subsecname}    

  \end{frame}

  \subsection{Résultats} %todo
  \begin{frame}{\subsecname}

  \end{frame}
  
%---------- Conslusion

  \section{Conclusion}
  \begin{frame}{\secname}
      \begin{block}{Rappels}
        \begin{itemize}
          \item Suivi de personnes
          \item Service web : réceptionner et traiter des données
          \item Client : fournir et demander des données
        \end{itemize}
      \end{block}

      \pause

      \begin{alertblock}{Points négatifs}
        \begin{itemize}
          \item Client uniquement android : iOs, Windows Universal, Site web
          \item Déploiement sur le cloud
        \end{itemize}
      \end{alertblock}

  \end{frame}

  \begin{frame}{\secname}
    \begin{exampleblock}{Points positifs}
        \begin{itemize}
          \item Architecture modulaire
          \item Processus d'intégration
          \begin{itemize}
            \item Intégration continue
            \item Tests automatiques
            \item Déploiement automatique
          \end{itemize}
        \end{itemize}
      \end{exampleblock}

      \pause

      \begin{block}{Perspectives}
        \begin{itemize}
          \item Projet accessible librement sur GitHub
          \item Tout le mode peut y contribuer
        \end{itemize}
      \end{block}
  \end{frame}

%---------- Remerciement

  \begin{frame}{Fin}
    \begin{center}
      Merci de votre attention.
    \end{center}
  \end{frame}

\end{document} %finished!
