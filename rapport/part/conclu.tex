%%%%%%%%%%%%%%%%%%%%%%%%%%%%%%%%%%%%%%%%%%%%%%%%%%%%%%%%%%%%%%%%%%%%%%%%
% CONCLUSION
%%%%%%%%%%%%%%%%%%%%%%%%%%%%%%%%%%%%%%%%%%%%%%%%%%%%%%%%%%%%%%%%%%%%%%%%

Ce projet a donc été l’occasion de concrétiser au travers de l’application WatchDogZZ une idée personnelle. Nous avons pu développer une application complète reprenant les principes fondamentaux de la carte du maraudeur à savoir la géolocalisation d’usagers dans un établissement. A ceci de nombreuses fonctionnalités ont pu être ajoutées comme le partage de position, la gestion de points d’intérêts, etc.
Ceci est rendu possible par le développement combiné d'une partie serveur et d'un client. Le client est une application Android compatible avec tout dispositif (smartphone, tablette, télévision, etc.) disposant d’un système en version 12 ou supérieur. Le web service est basé sur le framework NodeJS couplé à une base de données NoSQL : MongoDB ; permettant la communication de données sécurisées par le protocole HTTPS. Ce sont donc des technologies émergente qui sont utilisées dans cette solution.

Au terme de ce projet tous les objectifs initiaux ont été atteints et de nombreuses fonctionnalités supplémentaires et de concepts originaux restent à développer. La taille du projet, ses spécifications et sa pluralité technologique en font un projet très complexe et demanderait beaucoup plus de temps pour être complet.
C'est pourquoi, en début de projet, il a été décidé de se concentrer sur les fonctionnalités de base du service que nous souhaitions implémenter, afin de disposer d'une base solide et évolutive.
Ceci a ensuite permis d’implémenter des fonctionnalités plus complexes dans des itérations de type "agile" en assurant qu’au terme du projet nous aurions une application fonctionnelle et répondant aux critères initiaux.
C'est ce cheminement qui a permis la création de la solution WatchDogZZ, fonctionnelle, et disposant de certaines fonctionnalités supplémentaires.

Ce projet nous a permis d’atteindre quatre objectifs que nous nous étions fixés. Les deux premiers étaient des objectifs techniques à savoir progresser dans la maîtrise d’Android et NodeJS afin d’augmenter nos atouts techniques sur le marché du travail. Le troisième était de produire une application dans la lignée des applications modernes c’est-à-dire interagissant avec des services web et proposant des interactions sociales. Enfin le dernier objectif qui fut proposé par notre tuteur et qui consistait à progresser dans nos méthodes de génie logiciel en utilisant de l’intégration continu et du développement agile. L’atteinte de ses objectifs relève d’une plus grande satisfaction que la production de la solution elle-même.

De nombreux axes d’amélioration et d’évolution restent encore ouverts pour notre projet. Tout d’abord, toutes les fonctionnalités auxquelles nous avons pensé n’ont pas toutes été implémentées comme par exemple la vue en réalité virtuelle de l’établissement ou encore le calcul d’itinéraire. Celles-ci restent facilement intégrables dans l’application qui possèdent tous les prérequis à leur intégration.

Du point de vue de la diffusion de l'application, deux points majeurs peuvent être améliorés. Le premier concerne la portabilité du client : en effet il n’est actuellement disponible que sur les appareils Android, un portage iOS et Windows Phone permettrait de cibler la quasi-totalité du marché mobile. Le second point est le passage à l’échelle du web service : les tests exécutés montrent que pour un nombre très faible d’utilisateurs les performances du service sont parfaites, cependant pour un nombre d’usagers plus important le comportement de notre solution nous est encore inconnu. Ces améliorations potentielles pourraient faire de WatchDogZZ une application complète et sérieuse pouvant satisfaire les cas réels présentés dans ce rapport.
