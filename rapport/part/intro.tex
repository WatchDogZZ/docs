%%%%%%%%%%%%%%%%%%%%%%%%%%%%%%%%%%%%%%%%%%%%%%%%%%%%%%%%%%%%%%%%%%%%%%%%
% INTRODUCTION
%%%%%%%%%%%%%%%%%%%%%%%%%%%%%%%%%%%%%%%%%%%%%%%%%%%%%%%%%%%%%%%%%%%%%%%%

Ce projet s'inscrit dans la 3\up{ème} année du cycle ingénieur à l’Institut Supérieur d’Informatique de Modélisation et de leurs Applications (ISIMA) et s'étend sur une durée de 120h par personne. L'idée du sujet est de notre propre initiative et se base sur une constatation : il est parfois difficile de s'orienter dans un bâtiment de grande taille.

L'objectif est donc de proposer et mettre en place une solution évolutive, innovante est pratique pour les utilisateurs afin de s'orienter dans les bâtiments de l'ISIMA. Nous pouvons penser que ce type de solution peut s'étendre à tout bâtiment au sein duquel il est autorisé d'être localisé. Cette solution peut avoir des applications dans le domaine du secourisme, ce qui peut permettre aux sapeurs pompiers de localiser des personnes dans un bâtiment enfumé facilement, permettre à des entreprises hébergeant des données sensible de localiser ses visiteurs ou collaborateurs ou encore optimiser les déplacement de personnes dans des bâtiments de grande taille (orientation de médecins dans un hôpital, de techniciens dans une usine).

Nous verrons que la solution mise en place s'organise en 2 projets. Le premier projet consistera en un Service Web capable de répondre à des requêtes utilisateur de type \underline{http}. Ces requêtes permettront aux utilisateurs d'envoyer leur position afin de la stocker sur le serveur et de l'envoyer aux autres utilisateurs qui en font la demande. Le second projet consistera en la création d'une application cliente de ce Service Web afin d'en illustrer le fonctionnement. La plateforme choisie pour le développement du client est Android.

L'étude débutera par une partie de présentation plus précise du sujet, ainsi qu'une étude préalable à la réalisation de la solution. Ensuite, la partie d'implémentation des deux projets sera détaillée pour être enfin testée et discutée dans une dernière partie.

