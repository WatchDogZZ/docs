% -------------------------------------------------------------------- %
%                                                                      %
% fichier de préambule                                                 %
%                                                                      %
% Importantion de nombreux packages, plus francisation                 %
%                                                                      %
% -------------------------------------------------------------------- %

% ----------------------------------------------------------------------
% gestion du français, et des accents
\usepackage{cmap}
\usepackage[T1]{fontenc}
\usepackage[english,frenchb]{babel}
\usepackage[utf8]{inputenc}
\usepackage{helvet}
\usepackage{courier}
%\renewcommand{\familydefault}{\sfdefault}
%\pdfcompresslevel=3
%\usepackage{fourier}

% \usepackage{pdfpages} % pour inclure des pdf directement

\usepackage{datetime} % pour avoir l'heure de compilation avec \currenttime

\usepackage{pdflscape} % page en paysage

\usepackage{color}
\usepackage[table]{xcolor}

\usepackage{sectsty}
\allsectionsfont{\bfseries\sffamily\color{black!90}\hspace{-1em}}
\partfont{\sffamily}
\usepackage[font={small,it}]{caption}
\makeatletter
\def\@seccntformat#1{\llap{\csname the#1\endcsname\quad}}
\makeatletter\pdfcompresslevel=9
\def\@seccntformat#1{\llap{\csname the#1\endcsname\quad}}
\makeatother

% ----------------------------------------------------------------------
% bibliography
\usepackage[backend=biber, bibencoding=utf8 , sorting=none,hyperref=true,url=false,backref=true,backrefstyle=three]{biblatex}
\bibliography{input/biblio}

% ----------------------------------------------------------------------
% packages pour les figures
\usepackage{graphicx}
\usepackage{subfig}
\usepackage{tikz}

% change la numéroration des figures
%%%% debut macro %%%%
\makeatletter
\renewcommand{\thefigure}{\ifnum \c@section>\z@ \thesection.\fi
 \@arabic\c@figure}
\@addtoreset{figure}{section}
\makeatother
%%%% fin macro %%%%

% change la numéroration des tableaux
%%%% debut macro %%%%
\makeatletter
\renewcommand{\thetable}{\ifnum \c@section>\z@ \thesection.\fi
 \@arabic\c@table}
\@addtoreset{table}{section}
\makeatother
%%%% fin macro %%%%

% ----------------------------------------------------------------------
% diagramme de Gantt
%\usepackage{model/pgfgantt}

% ----------------------------------------------------------------------
% pour mettre une page en mode paysage
\usepackage{lscape}

% ----------------------------------------------------------------------
% Miscelianous
\usepackage[pdfborder={0 0 0 [3 3]},pdftex,unicode=true]{hyperref}
\usepackage{bookmark}

%\usepackage{tablefootnote}

% ----------------------------------------------------------------------
%packages pour les maths
\usepackage{amsmath}
\usepackage{amssymb}
\usepackage{mathabx} %   \ldbrack et \rdbrack
%\usepackage{amsfonts}
\usepackage{mathrsfs}
\usepackage{wasysym}
\usepackage{textcomp}
%\usepackage{bbm}
\DeclareMathAlphabet{\mathpzc}{OT1}{pzc}{m}{it} % \mathpzc{ABCdef}
\usepackage{eufrak}                             % \mathfrak{ABCdef}

% change la numéroration des équations
\numberwithin{equation}{section}

% ----------------------------------------------------------------------
% pour les unités
%\usepackage[load=physical]{siunitx}
\newcommand{\eV}{\mathrm{eV}}
\newcommand{\s}{\mathrm{s}}
\newcommand{\m}{\mathrm{m}}
\newcommand{\g}{\mathrm{g}}
\newcommand{\V}{\mathrm{V}}
\newcommand{\sample}{\mathrm{s}}
\newcommand{\Hz}{\mathrm{Hz}}
\newcommand{\pc}{\mathrm{pc}}

\newcommand{\nano}{\mathrm{n}}
\newcommand{\micro}{\mathrm{\mu}}
\newcommand{\milli}{\mathrm{m}}
\newcommand{\centi}{\mathrm{c}}
\newcommand{\kilo}{\mathrm{k}}
\newcommand{\mega}{\mathrm{M}}
\newcommand{\giga}{\mathrm{G}}
\newcommand{\tera}{\mathrm{T}}
\newcommand{\exa}{\mathrm{E}}
\newcommand{\zetta}{\mathrm{Z}}


% ----------------------------------------------------------------------
% packages pour les algorithmes
\usepackage[section]{algorithm}
\usepackage[noend]{algpseudocode}

\renewcommand{\algorithmiccomment}[1]{{\hfill$\vartriangleright$\scriptsize\sffamily~ #1}}

% francisation des algorithmes
\floatname{algorithm}{\sffamily Algorithme}

\renewcommand{\algorithmicprocedure} {{\footnotesize \textbf{\textsf{Proc\'edure}} }}
\renewcommand{\algorithmicwhile}     {{\footnotesize \textbf{\textsf{Tant que}}    }}
\renewcommand{\algorithmicdo}        {{\footnotesize \textbf{\textsf{Faire}}       }}
\renewcommand{\algorithmicend}       {{\footnotesize \textbf{\textsf{Fin}}         }}
\renewcommand{\algorithmicif}        {{\footnotesize \textbf{\textsf{Si}}          }}
\renewcommand{\algorithmicelse}      {{\footnotesize \textbf{\textsf{Sinon}}       }}
\renewcommand{\algorithmicthen}      {{\footnotesize \textbf{\textsf{Alors}}       }}
\renewcommand{\algorithmicfor}       {{\footnotesize \textbf{\textsf{Pour}}        }}
\renewcommand{\algorithmicforall}    {{\footnotesize \textbf{\textsf{Pour tout}}   }}
\renewcommand{\algorithmicdo}        {{\footnotesize \textbf{\textsf{Faire}}       }}
\renewcommand{\algorithmicrepeat}    {{\footnotesize \textbf{\textsf{Répéter}}     }}
\renewcommand{\algorithmicuntil}     {{\footnotesize \textbf{\textsf{Jusqu'à}}     }}
\renewcommand{\algorithmicfunction}  {{\footnotesize \textbf{\textsf{Fonction}}    }}
\renewcommand{\algorithmicreturn}    {{\footnotesize \textbf{\textsf{Retourner}}   }}
\let\mylistof\listof
\renewcommand\listof[2]{\mylistof{algorithm}{Liste des algorithmes}}
\makeatletter
\providecommand*{\toclevel@algorithm}{0}
\makeatother
% Lister les algorithmes :
%\listofalgorithms % même principe que toc
\addto\captionsfrench{%
  \renewcommand{\listfigurename}{Liste des figures}%
}
\addto\captionsfrench{\def\figurename{Figure}}
\addto\captionsfrench{\def\tablename{Tableau}}

% ----------------------------------------------------------------------
% tableaux
\definecolor{tablegray}{gray}{0.8}%table

% ----------------------------------------------------------------------
% importantion de code
\definecolor{mygray}{rgb}{0.5,0.5,0.5} % définition du gris (n° ligne)
\usepackage{listings}
\usepackage{listingsutf8}

% couleurs pour listings
\definecolor{lightgray}{rgb}{.9,.9,.9}
\definecolor{darkgray}{rgb}{.4,.4,.4}
\definecolor{purple}{rgb}{0.65, 0.12, 0.82}

% javascript definition
\lstdefinelanguage{Javascript}{
  keywords={typeof, new, true, false, catch, function, return, null, catch, switch, var, if, in, while, do, else, case, break},
  keywordstyle=\color{blue}\bfseries,
  ndkeywords={class, export, boolean, throw, implements, import, this},
  ndkeywordstyle=\color{darkgray}\bfseries,
  identifierstyle=\color{black},
  sensitive=false,
  comment=[l]{//},
  morecomment=[s]{/*}{*/},
  commentstyle=\color{purple}\ttfamily,
  stringstyle=\color{red}\ttfamily,
  morestring=[b]',
  morestring=[b]"
}

% pour la numerotation des codes
\usepackage{chngcntr}
\AtBeginDocument{\counterwithin{lstlisting}{section}}

\renewcommand{\lstlistlistingname}{Liste des extraits de code}
%%options for listings
\DeclareCaptionFont{white}{\color{white}}
\DeclareCaptionFormat{listing}{\colorbox{black!80}{\parbox{\textwidth}{#1#2#3}}}
\captionsetup[lstlisting]{format=listing,labelfont={white,tt},textfont=white}
\definecolor{Rred}{rgb}{0.6,0,0} % for strings
\definecolor{Rgreen}{rgb}{0.25,0.5,0.35} % comments
\definecolor{Rpurple}{rgb}{0.5,0,0.35} % keywords
\definecolor{Rdocblue}{rgb}{0.25,0.35,0.75} % Rdoc
\lstset{
    float,
    columns=fullflexible,
    numbers=left,
    tabsize=2,
    breaklines=true,
    basicstyle=\small\ttfamily,
    basewidth=0.51em,
    showspaces=false,
    showstringspaces=false,
    stringstyle=\color{Rred},
    commentstyle=\color{Rgreen},
    keywordstyle=\color{Rdocblue}
}
% gestion des accents dans le code
\renewcommand{\lstlistingname}{Code}
\lstset{literate=
    {á}{{\'a}}1 {é}{{\'e}}1 {í}{{\'i}}1 {ó}{{\'o}}1 {ú}{{\'u}}1
    {Á}{{\'A}}1 {É}{{\'E}}1 {Í}{{\'I}}1 {Ó}{{\'O}}1 {Ú}{{\'U}}1
    {à}{{\`a}}1 {è}{{\`e}}1 {ì}{{\`i}}1 {ò}{{\`o}}1 {ù}{{\`u}}1
    {À}{{\`A}}1 {È}{{\'E}}1 {Ì}{{\`I}}1 {Ò}{{\`O}}1 {Ù}{{\`U}}1
    {ä}{{\"a}}1 {ë}{{\"e}}1 {ï}{{\"i}}1 {ö}{{\"o}}1 {ü}{{\"u}}1
    {Ä}{{\"A}}1 {Ë}{{\"E}}1 {Ï}{{\"I}}1 {Ö}{{\"O}}1 {Ü}{{\"U}}1
    {â}{{\^a}}1 {ê}{{\^e}}1 {î}{{\^i}}1 {ô}{{\^o}}1 {û}{{\^u}}1
    {Â}{{\^A}}1 {Ê}{{\^E}}1 {Î}{{\^I}}1 {Ô}{{\^O}}1 {Û}{{\^U}}1
    {œ}{{\oe}}1 {Œ}{{\OE}}1 {æ}{{\ae}}1 {Æ}{{\AE}}1 {ß}{{\ss}}1
    {ç}{{\c c}}1 {Ç}{{\c C}}1 {ø}{{\o}}1 {å}{{\r a}}1 {Å}{{\r A}}1
    {€}{{\EUR}}1 {£}{{\pounds}}1
}

% ----------------------------------------------------------------------
% trucs en vracs
\usepackage[         %
    top    = 2.75cm, %
    bottom = 3.50cm, %
    left   = 3.00cm, %
    right  = 2.50cm]{geometry}   % marges
\usepackage{makeidx}             % make index (finalement je ne l'utilise pas)
\makeindex                       % génération de l'index, besoin d'une compilation séparée

\usepackage{fancyhdr}            % entête et pied de page (pour les modifier)
\pagestyle{fancy}                % style des pages
\fancyhf{}
\renewcommand{\sectionmark}[1]{\markboth{\thesection{} -- #1}{}}
\fancyhead[L]{\leftmark}
\fancyhead[R]{\thesubsection}
\makeatletter
\fancyfoot[L]{\@author \space --- \schoolShortName{} }
\makeatother 
\fancyfoot[R]{\thepage}
\renewcommand{\footrulewidth}{0.1pt}

\usepackage{setspace}     % gestion des interlignages
\onehalfspacing           % interlignage de 1.5

% ----------------------------------------------------------------------
%Page de titre
\newcommand{\HRule}{\rule{\linewidth}{0.5mm}}
\usepackage{model/titlepage}

% ----------------------------------------------------------------------
% modification du style des parties (pour la classe article)
\makeatletter
\renewcommand\part{%
  \clearpage
  \thispagestyle{empty}%
  \null\vfill
  \secdef\@part\@spart}
\def\@part[#1]#2{%
    \ifnum \c@secnumdepth >-2\relax
      \refstepcounter{part}%
      \addcontentsline{toc}{part}{\thepart\hspace{1em}#1}%
    \else
      \addcontentsline{toc}{part}{#1}%
    \fi
    \markboth{}{}%
    {\centering
     \interlinepenalty \@M
     \vskip 20\p@
     \normalfont
     \ifnum \c@secnumdepth >-2\relax
       \huge\bfseries \partname\nobreakspace\thepart
       \par
       \vskip 20\p@
     \fi
     \Huge \bfseries\sffamily #2\par}%
    \@endpart}
\def\@spart#1{%
    {\centering
     \interlinepenalty \@M
     \vskip 20\p@
     \normalfont
     \Huge \bfseries #1\par}%
    \@endpart}
\def\@endpart{\vfill
              \newpage
              }
\makeatother


\renewcommand*{\thefootnote}{(\arabic{footnote})}
% ----------------------------------------------------------------------
% nouvelles commandes
\newcommand{\fraction}[2]{\raisebox{0.5ex}{#1}\slash\raisebox{-0.5ex}{#2}}
% pour afficher 1/2, ne fonctionne pas dans un environnement mathématique

\newcommand{\ie}{\emph{i.e.}}
\newcommand{\eg}{\emph{e.g.}}

\newcommand{\Cpp}{\emph{C++}}
\newcommand{\Python}{\emph{Python}}
\newcommand{\ROOT}{\texttt{ROOT}}

\newcommand{\includeMd}[1]{\input{|"pandoc --wrap=preserve --listings -t latex #1"}}
\def\tightlist{}
