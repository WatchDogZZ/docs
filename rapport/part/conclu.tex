%%%%%%%%%%%%%%%%%%%%%%%%%%%%%%%%%%%%%%%%%%%%%%%%%%%%%%%%%%%%%%%%%%%%%%%%
% CONCLUSION
%%%%%%%%%%%%%%%%%%%%%%%%%%%%%%%%%%%%%%%%%%%%%%%%%%%%%%%%%%%%%%%%%%%%%%%%

Ce projet a donc été l’occasion de concrétiser au travers de l’application WatchDogZZ notre idée personnelle. Nous avons pu développer une application complète reprenant les principes fondamentaux de la carte du maraudeur à savoir la géolocalisation d’usagers dans un établissement. A ceci de nombreuses fonctionnalités ont pu être ajoutées comme le partage de position, la gestion de points d’intérêts, etc.

Ceci est rendu possible par le développement intégral des parties client et serveur. Le client est une application Android compatible avec tout dispositif (smartphone, tablette, télévision, etc.) disposant d’un système en version 12 ou supérieur. Le web service est basé sur la technologie nodeJS couplé à une base de données NoSQL MongoDB permettant la communication de données sécurisées par le protocole HTTP.

Au terme de ce projet tous les objectifs initiaux ont été atteints et de nombreuses fonctionnalités supplémentaires et de concepts originaux restent à développer. La taille du projet, ses spécifications et sa pluralité technologique en font un projet très complexe et demanderait beaucoup plus de temps pour être complet. Devant ce constat le choix judicieux a été fait en début de projet de se concentrer dans un premier temps à produire une solution primaire, élémentaire et avec seulement les fonctionnalités de bases afin d’avoir un livrable fonctionnel. Ceci a ensuite permis d’implémenter des fonctionnalités plus complexes dans des itérations agiles en assurant qu’au terme du projet nous aurions une application fonctionnelle et répondant aux critères initiaux. Ainsi nous avons pu produire WatchDogZZ avec certaines fonctionnalités supplémentaires.

De nombreux axes d’amélioration et d’évolution restent encore ouverts pour notre projet. Tout d’abord, toutes les fonctionnalités auxquelles nous avons pensé n’ont pas toutes été implémentées comme par exemple la vue en réalité virtuelle de l’établissement ou encore le calcul d’itinéraire. Cependant celles-ci restent facilement intégrables dans l’application qui possèdent tous les prérequis à leur intégration. D’un point de vue plus diffusion de l’application, deux points majeurs peuvent être améliorés. Le premier est la portabilité du client : en effet il n’est actuellement disponible que sur les appareils Android, le porter sur iOS et Windows Phone permettrait de pouvoir toucher la quasi-totalité du marché cible. Le deuxième est le passage à l’échelle du web service : les tests exécutés montrent que pour un nombre très faible d’utilisateurs les performances du service sont parfaites mais concernant un nombre d’usagers plus important le comportement de notre solution nous est encore inconnu. Ces améliorations potentielles pourraient faire de WatchDogZZ une application complète et sérieuse pouvant satisfaire les cas réels présentés dans ce rapport.
