\subsubsection{Fonctionnalités du Service Web}

Requetes sur le web service
Des URL sont mises à disposition par le service et permettent d'effectuer certaines tâches. Le passage de paramètres pour ces URL se fait directement dans le corps de la requête sous forme de JSON.
Les réponses du service sont sous forme d'objet JSON dans le corps de la réponse. Les réponses contiennent les champs suivants :

\lstset{language=Javascript}
\begin{lstlisting}[caption=Corps de la réponse serveur]
{
    'status': 'ok' / 'fail', // L'état de la requête
    'error': 'description' // Une description de l'erreur s'il y en a une
}
\end{lstlisting}


La plupart des requêtes décrites nécéssitent que l'utilisateur soit authentifié.
/login
La première requête à effectuer sur le service doit s'effectuer avec la méthode POST sur cette URL. Pour se connecter, l'utilisateur doit envoyer les paramètres suivants: 
\lstset{language=Javascript}
\begin{lstlisting}[caption=Corps de la requête login]
{
    'name': 'username', // Le nom de l'utilisateur à connecter
    'location': [1.0, 2.0, 3.0] // La position courrante de l'utilisateur
}
\end{lstlisting}

/who
Cette URL permet de récupérer en méthode GET une liste de noms de personnes actuellement en ligne. Le retour est sous la forme suivante:
\lstset{language=Javascript}
\begin{lstlisting}[caption=Corps de la requête who]
{
    'list': [
        'userName1',
        'userName2',
        'userName3'
    ]
}
\end{lstlisting}

/where
En utilisant la méthode GET, cette URL retourne la liste des utilisateurs ainsi que leur position. En utilisant la méthode POST et en passant les paramètres adéquats, l'utilisateur peut mettre à jour sa position.
En méthode GET, le service renvoie une liste des utilisateurs connectés avec leurs positions :
\lstset{language=Javascript}
\begin{lstlisting}[caption=Corps de la requête where GET]
{
    'list': [
        {
            'name': 'username1',
            'location': [1.0, 2.0, 3.0]
        },
        {
            'name': 'username2',
            'location': [1.0, 2.0, 3.0]
        },
        {
            'name': 'username3',
            'location': [1.0, 2.0, 3.0]
        }
    ]
}
\end{lstlisting}

En utilisant la méthode POST, l'utilisateur met à jour sa position. Les paramètres à envoyer sont les suivants :

\lstset{language=Javascript}
\begin{lstlisting}[caption=Corps de la requête where POST]
{
    'name': 'username',
    'location': [1.0, 2.0, 3.0]
}
\end{lstlisting}

Si l'utilisateur n'est pas connecté au moment de mettre à jour sa position, le serveur va tenter de l'authentifier et sauvegarder sa position.
Partie administration
Les fonctionnalités minimales pour l'application d'administration sont :
\begin{itemize}
    \item visualisation des logs du serveur
    \item visualisation du contenu de la base de donnée
\end{itemize}

Ensuite les fonctionnalités avancées pourront être :
\begin{itemize}
    \item gestion des utilisateurs
\end{itemize}
