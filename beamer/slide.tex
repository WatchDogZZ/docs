%\documentclass[draft]{beamer} %temporary render
\documentclass{beamer} %final render

\usepackage[french]{babel} 	%utilisation des
\usepackage[T1]{fontenc} 	%caracteres
\usepackage[utf8]{inputenc} 	%francais
\usepackage{graphicx}
\usepackage{color}
\usepackage{ccaption}
\usepackage{fancyvrb}
\usepackage{verbatim}
\usepackage{float}
\usepackage{csquotes}	%pour les citations
\usetheme{Frankfurt}
\usepackage{marvosym} % \MVRIGHTarrow

\AtBeginSection[] 	%va permettre fl'affichage du sommaire avant
{					%chaque vouvelle partie
  \begin{frame}
    \frametitle{Plan}
    \tableofcontents[currentsection, hideothersubsections]
  \end{frame} 
}

\setbeamertemplate{navigation symbols}{}
\setbeamertemplate{footline}[frame number]
\setbeamercolor{footline}{fg=gray,bg=black}

\makeatletter
\newcommand\titlegraphicii[1]{\def\inserttitlegraphicii{#1}}
\titlegraphicii{}
\setbeamertemplate{title page}
{
  \vbox{}
   {\usebeamercolor[fg]{titlegraphic}\inserttitlegraphic\hfill\inserttitlegraphicii\par}
  \begin{centering}
    \begin{beamercolorbox}[sep=8pt,center]{institute}
      \insertinstitute
    \end{beamercolorbox}
    \begin{beamercolorbox}[sep=8pt,center]{title}
      \usebeamerfont{title}\inserttitle\par%
      \ifx\insertsubtitle\@empty%
      \else%
        \vskip0.25em%
        {\usebeamerfont{subtitle}\usebeamercolor[fg]{subtitle}\insertsubtitle\par}%
      \fi%     
    \end{beamercolorbox}%
    \vskip1em\par
    \begin{beamercolorbox}[sep=8pt,center]{author}
      \usebeamerfont{author}\insertauthor
    \end{beamercolorbox}
    \begin{beamercolorbox}[sep=8pt,center]{date}
      \usebeamerfont{date}\insertdate
    \end{beamercolorbox}%\vskip0.5em
  \end{centering}
  %\vfill
}
\makeatother
\author{Benjamin BARBESANGE et Benoît GARCON}
\title{Soutenance de Projet Ingénieur 3ème Année}
\subtitle{WatchDogZZ - Suivi de personnes dans les bâtiments}
%\institute{Siemens Industry Software\\ISIMA}
\date{Projet de 120h}
% \titlegraphicii{\includegraphics[height=5mm]{watchdogzz.png}}
\titlegraphic{\includegraphics[height=5mm]{isima.png}}

\logo{\includegraphics[height=10mm]{logo.png}} %le logo en bas a droite

\begin{document}

	\begin{frame} %frame de titre
		\maketitle
    \vspace*{1cm}
    \footnotesize
    \begin{tabular}{ll}
      Tuteur projet &: Pierre COLOMB\\
      Référent ISIMA &: Eva HASSINGER
    \end{tabular}
	\end{frame}

%---------- Introduction

  \begin{frame}{Introduction}
    
    \begin{block}{Contexte}
      \begin{itemize}
        \item Proposition personnelle
        \item Domaine du tracking
        \item Nouvelles technologies
      \end{itemize}
    \end{block}

    \pause

    \begin{exampleblock}{Applications}
      \begin{itemize}
        \item Secourisme
        \item Sécurité
        \item Optimisation de déplacements
        \item Analyser des déplacements dans un bâtiment
      \end{itemize}
    \end{exampleblock}

    \pause

    \begin{alertblock}{Objectifs}
      \begin{itemize}
        \item Suivre des personnes en \textbf{temps réel}
        \item Architecture \textbf{simple} et \textbf{modulaire}
      \end{itemize}
    \end{alertblock}

  \end{frame}

%---------- Plan

  \begin{frame}{Plan}
    \tableofcontents
  \end{frame}

%---------- Etude

  \section{Présentation du projet}
  \begin{frame}{\secname}
    \begin{center}
      Carte du Marauder Harry Potter.
    \end{center}

    \begin{block}{Objectifs}
      \begin{itemize}
        \item Visualiser un bâtiment
        \item Visualiser les personnes
        \item Opérations auxiliaires
        \begin{itemize}
          \item Partager sa position
          \item Marqueurs personnalisés
          \item Itinéraires
        \end{itemize}
        \item Données à caractère personnel
      \end{itemize}
    \end{block}
    
    \begin{alertblock}{Contraintes}
      \begin{itemize}
        \item Solution évolutive
        \item Implémentation d'une base
        \item Constante amélioration (non régression)
      \end{itemize}
    \end{alertblock}
    
  \end{frame}


  \subsection{Analyse de l'existant}
  \begin{frame}{\subsecname}
    \begin{block}{Méthodes de localisation}
      \begin{itemize}
        \item Intentionnelle
        \item Automatique
        \item \textbf{Open data}
      \end{itemize}  
    \end{block}
    
    \begin{block}{Géolocalisation}
      \begin{itemize}
        \item GPS intégré
        \item Smart*, Google Glass, 
        \item Suit les déplacements de l'utilisateur
      \end{itemize}
    \end{block}
    
  \end{frame}

  \subsection{Architecture proposée}
  \begin{frame}{\subsecname}

    \begin{columns}
      \begin{column}{0.5\textwidth}
        \begin{exampleblock}{Objectif}
          \begin{itemize}
            \item Implémenter un service web
            \item Architecture en 2 parties
            \begin{itemize}
              \item Client mobile
              \item Service web
            \end{itemize}
          \end{itemize}
        \end{exampleblock}
      \end{column}
      \begin{column}{0.5\textwidth}
        \begin{figure}
        \includegraphics[width=\linewidth, height=\textheight, keepaspectratio]{archi_finale.png}
        \caption{Architecture proposée}
        \end{figure}
      \end{column}
    \end{columns}

    \begin{block}{Répartition des tâches}
      \begin{itemize}
        \item Benoît : Client mobile
        \item Benjamin : Service web
      \end{itemize}
    \end{block}
  \end{frame}

  \subsection{Spécifications}
  \begin{frame}{\subsecname}
    \begin{block}{Service web}
      \begin{itemize}
        \item REST
        \item Stocke et distribue des données
      \end{itemize}
    \end{block}

    \begin{alertblock}{Fonctionnalités requises}
      \begin{itemize}
        \item Connexion / déconnexion du service
        \item Fournir la liste des personnes connectées
        \item Fournir la position des utilisateurs
      \end{itemize}
    \end{alertblock}

    \begin{exampleblock}{Fonctionnalités supplémentaires}
      \begin{itemize}
        \item Historique de positions
        \item Calcul d'itinéraires
        \item Partage de position (sms, mail)
      \end{itemize}
    \end{exampleblock}

  \end{frame}

  \begin{frame}{\subsecname}
    \begin{block}{Android}
      \begin{itemize}
        \item \textbf{Un client} du service
        \item Mobile : \textbf{portabilité}, choix personnel de développement
        \item But : servir d'\textbf{interface} aux utilisateurs finaux
      \end{itemize}
    \end{block}

    \begin{alertblock}{Fonctionnalités requises}
      \begin{itemize}
        \item Gérer un utilisateur
        \item Consommer le service
        \item Répondre à des critères d'\textbf{utilisabilité} et de \textbf{performances}
      \end{itemize}
    \end{alertblock}

    \begin{exampleblock}{Fonctionnalités supplémentaires}
      \begin{itemize}
        \item Visualisation d'une carte en 3D
        \item Visualisation en réalité virtuelle
        \item Ajout d'informations personnalisées
      \end{itemize}
    \end{exampleblock}

  \end{frame}


  \section{Réalisation de la solution}
  \subsection{Technologies transverses}
  \begin{frame}{\subsecname}
    \begin{block}{GitHub}
      \begin{itemize}
        \item \textbf{Versionner} le code source
        \item \textbf{Planifier} et \textbf{assigner} de tâches
        \item Recenser les bugs
      \end{itemize}
    \end{block}

    \begin{block}{Travis CI}
      \begin{itemize}
        \item Intégration continue
        \item \textbf{Tests} et \textbf{déploiements} automatiques
      \end{itemize}
    \end{block}
    
  \end{frame}

  \subsection{Technologies mobile}
  \begin{frame}{\subsecname}
  \begin{columns}
    \begin{column}{0.58\textwidth}
      \begin{block}{Android}
        \begin{itemize}
          \item Android studio
          \item Framework (version > 12)
          \item Gestion du réseau
          \begin{itemize}
            \item Volley
            \item Picasso
          \end{itemize}
          \item Bibliothèque graphique : OpenGL ES
        \end{itemize}
      \end{block}
      
    \end{column}
    \begin{column}{0.38\textwidth}
      \begin{figure}
        \includegraphics[height=0.25\textheight, keepaspectratio]{android.png}
      \end{figure}
      \begin{figure}
        \includegraphics[width=\linewidth, height=\textheight, keepaspectratio]{opengl.png}
      \end{figure}
    \end{column}
  \end{columns}
    
  \end{frame}

  \subsection{Réalisation mobile}
  \begin{frame}{\subsecname}
      \begin{center}
        Architecture MVC
        \begin{figure}
        \includegraphics[width=0.58\linewidth, height=\textheight, keepaspectratio]{android-model.png}
        \end{figure}
      \end{center}

      \vspace*{-8mm}

      \begin{columns}
        \begin{column}{0.58\textwidth}
          \begin{figure}
          \includegraphics[width=\linewidth, height=\textheight, keepaspectratio]{android-controller.png}
          \end{figure}
        \end{column}
        \begin{column}{0.58\textwidth}
          \begin{figure}
          \includegraphics[width=\linewidth, height=\textheight, keepaspectratio]{android-model.png}
          \end{figure}
        \end{column}
      \end{columns}

  \end{frame}

  \begin{frame}{\subsecname}
    \begin{block}{Composition de l'application}
      \begin{itemize}
        \item 3 activités
        \begin{itemize}
          \item Unité d'action
          \item Des fragments
        \end{itemize}
      \end{itemize}
    \end{block}

    \begin{block}{Communications}
      \begin{itemize}
        \item Utilisation de services asynchrones
        \item Communication réseau
      \end{itemize}
    \end{block}
    
  \end{frame}

  \subsection{Technologies serveur}
  \begin{frame}{\subsecname}

    \begin{columns}
      \begin{column}{0.58\textwidth}
        \begin{block}{Technologies}
          \begin{itemize}
            \item Framework NodeJS
            \item Ajout de paquets NPM
            \begin{itemize}
              \item Express : serveur web
              \item Jasmine : tests par specs
              \item MongoDB : gestion de la base
              \item Winston : système de log
            \end{itemize}
            \item MongoDB : base de données
          \end{itemize}
        \end{block}
        
      \end{column}
      \begin{column}{0.38\textwidth}
        \begin{figure}
        \includegraphics[width=\linewidth, height=\textheight, keepaspectratio]{nodeexpress.png}
        \end{figure}
        \begin{figure}
          \includegraphics[width=\linewidth, height=0.1\textheight, keepaspectratio]{npm.png}
        \end{figure}
      \end{column}
    \end{columns}

    \begin{exampleblock}{Services ajoutés}
      \begin{itemize}
        \item DNS et DynDNS
        \item Https avec Letsencrypt
        \item Log d'erreurs
      \end{itemize}
    \end{exampleblock}

  \end{frame}

  \subsection{Réalisation serveur}
  \begin{frame}{\subsecname}
    \vspace*{-2mm}
    \begin{figure}
      \includegraphics[width=0.8\linewidth, height=\textheight, keepaspectratio]{architecture-service-web-simple.png}
    \end{figure}
    \begin{block}{Description}
      \begin{itemize}
        \item server.js : code \textbf{serveur} et gestion de \textbf{requêtes}
        \item service.entities.js : gestion des \textbf{entités} et \textbf{valeurs par défaut}
        \item service.methods.js : wrapper l'utilisation de la base de données
        \item database.js : code de gestion \textbf{MongoDB}
      \end{itemize}
    \end{block}
  \end{frame}

  \section{Résultats}
  \subsection{Service à disposition}
  \begin{frame}{\subsecname}
    Plusieurs URLs:
    /GET
    Who : liste des personnes connectées
    Where : positions des personnes connectées et informations supplémentaires
    /POST
    Login : se connecter
    Logout : se déconnecter
    Where : mettre à jour sa position
    Exemple de paramètres pour une requête (/POST where){
        "token": "1A2Z3E4R5T6Y7U8I9O0P",
        "location": [
            1.0,
            2.0,
            3.0
        ]
    }  
  \end{frame}

  \subsection{Client Android}
  \begin{frame}{\subsecname}
    Un client potentiel du service.

    Localisation
    Affichage des utilisateurs connectés
    Synchronisation de la position avec le serveur
    Définition de points d’intérêt
    Affichage 3D

    Communication
    Authentification Google
    Partage de position
    22

    Fonctionnement
    Authentification
    Parcours de la carte
    Navigation dans le menu
    
  \end{frame}

  \subsection{Améliorations possibles}
  \begin{frame}{\subsecname}
    Client
      Amélioration de l’interface du client
      Mode réalité virtuelle
        Google cardboard
      Gestion des étages
        Calibration

    Serveur
      Calcul d’itinéraires
      Gestion des logs, alertes automatiques (erreurs graves)
    
  \end{frame}
  
%---------- Conclusion

  \section{Conclusion}
  \begin{frame}{\secname}
      \begin{block}{Rappels}
        \begin{itemize}
          \item Suivi de personnes
          \item Service web : réceptionner et traiter des données
          \item Client : fournir et demander des données
        \end{itemize}
      \end{block}

      \pause

      \begin{alertblock}{Points négatifs}
        \begin{itemize}
          \item Client uniquement android : iOs, Windows Universal, Site web
          \item Déploiement sur le cloud
        \end{itemize}
      \end{alertblock}

  \end{frame}

  \begin{frame}{\secname}
    \begin{exampleblock}{Points positifs}
        \begin{itemize}
          \item Architecture modulaire
          \item Processus d'intégration
          \begin{itemize}
            \item Intégration continue
            \item Tests automatiques
            \item Déploiement automatique
          \end{itemize}
        \end{itemize}
      \end{exampleblock}

      \pause

      \begin{block}{Perspectives}
        \begin{itemize}
          \item Projet accessible librement sur GitHub
          \item Tout le mode peut y contribuer
        \end{itemize}
      \end{block}
  \end{frame}

%---------- Remerciement

  \begin{frame}{Fin}
    \begin{center}
      Merci de votre attention.
    \end{center}
  \end{frame}

\end{document} %finished!
