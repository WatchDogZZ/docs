\section{Conception de la solution}

Dans cette section nous allons aborder plus précisément le déroulement de la conception de notre solution en détaillant nos travaux outils et méthodes.

\subsection{Conception du couple client/serveur}

Notre solution se décompose en effet en deux parties majeures et quasiment indépendante si l'on ne prend pas en compte l'interfaçage entre les deux. Nous allons donc voir en quoi celles-ci consistent exactement.

    \subsubsection{Web Service}

    Du coté serveur, il faudra gérer plusieurs parties. D'un coté, il faut s'occuper des requêtes utilisateur, et de l'autre, il faut stocker certaines données que l'on renverra aux utilisateurs.

    Pour ce faire, l'architecture est la suivante.

    \begin{figure}[H]
        \centering
        \includegraphics[width=\textwidth]{./img/architecture-service-web.png}
        \caption{Architecture du Service Web}
        \label{asweb}
    \end{figure}

    Comme nous pouvons l'observer dans la figure \ref{asweb}, blablabla.
    Le serveur + Mongodb.
    Pattern bridge pour la bdd. Des entités.

    \subsubsection{Application Android}
Le développement du client mobile a suivi le schéma classique de conception d’une application Android. Pour rappel les objectifs initiaux majeurs de ce client étaient de pouvoir récupérer des données sur un web service, les exploiter, les afficher à l’utilisateur et enfin retourner des données au service en question. De façon général la solution Android s’organise en deux projets directeurs : les tests et l’implémentation de l’application.

Il n’a pas été choisi ici de faire du développement dirigé par les tests car la technologie Android était dans le cadre de ce projet une découverte et il aurait était hasardeux de définir des tests Java sur les concepts Android. Les tests ont donc ici vocation à valider les mécaniques métiers a posteriori ainsi que le bon fonctionnement et l’intégrité de l’application au fur et à mesure de l’ajout de fonctionnalités. La partie relative aux tests se subdivise en deux autres : les tests unitaires et les tests instrumentés. Les premiers sont plutôt classiques et permettent de tester les mécaniques métiers et de vérifier tout ce qui est mockable, autrement dit simulable. Toutefois, il y a certains aspects dans un programme Android qu’il n’est pas possible de mocker sans enlever l’intérêt du test. Nous parlons ici de fonctionnalités s’appuyant intrinsèquement sur le système Android comme les appels réseaux, le GPS, l’écran, etc. Il est nécessaire dès lors que l’on veut simuler une fonctionnalité Android même basique, de simuler tout un système Android. D’où l’intérêt de la deuxième catégorie de tests : les tests instrumentés. Ceux-ci vont être exécutés sur un émulateur Android directement afin de pouvoir tester dans notre cas les appels réseaux et l’utilisation du GPS.

Le projet de tests est donc un projet annexe venant en soutien au projet principal : celui du développement de l’application Android. L’ensemble doit gérer des données et les afficher, c’est pourquoi un modèle MVC semblait adapté. Le modèle MVC se compose de trois parties en interaction comme c’est visible figure \ref{mvc}.

\begin{figure}[H]
    \centering
    \includegraphics[width=\textwidth]{./img/mvc.png}
    \caption{Patron de conception MVC}
    \label{mvc}
\end{figure}

Le modèle (M) représente les données traitées, dans le cas de l’application ce sont principalement les utilisateurs et leur position GPS. Les deux autres composants n’interagissent pas directement avec les données brutes mais ont accès à l’API du modèle symbolisée par le gestionnaire d’utilisateur (UserManager) comme sur la figure \ref{model}. Le modèle gère donc tous les petits traitements bas niveau sur les données et les sert au reste de l’architecture selon les besoins.

\begin{figure}[H]
    \centering
    \includegraphics[width=\textwidth]{./img/android-model.png}
    \caption{Diagramme du modèle}
    \label{model}
\end{figure}

Le contrôleur (C) s’occupe des interactions avec l’utilisateur, il doit pouvoir transmettre les commandes émanant de l’utilisateur aux autres composants. Dans le cadre de l’application Android, le contrôleur est symbolisé par les activités : ce sont les activités Android qui proposent les interfaces de contrôle utilisateur. Elles permettent de recevoir les évènements utilisateur comme un clic ou encore les messages du système comme l’arrivée d’un SMS ou encore l’actualisation de la position GPS. Les informations reçues peuvent ensuite être utilisées pour effectuer une mise à jour du modèle ou un rafraîchissement de l’affichage. Les différents types d’activités sont visible sur la figure \ref{controller}.

\begin{figure}[H]
    \centering
    \includegraphics[width=\textwidth]{./img/android-controller.png}
    \caption{Diagramme du contrôleur}
    \label{controller}
\end{figure}


La vue (V) est l’ensemble des éléments constituant la façon dont vont être présentées les données. La réalisation concrète dans l’application Android de la vue est faite par les fichiers XML de layout et autres ressources. Dans la vue entre aussi un pan important du concept de l’application : la gestion de l’affichage 3D. En effet, pour la plupart des éléments d’interface, tout peut être géré par des fichiers de métadonnées mais pour générer la carte en trois dimensions d’un lieu il est nécessaire de décrire dans le code la manière d’utiliser les données pour les afficher avec de véritables classes Java. La vue est donc un ensemble complexe constitué de diverses ressources comme on peut le voir sur la figure \ref{view}.

\begin{figure}[H]
    \centering
    \includegraphics[width=\textwidth]{./img/android-view.png}
    \caption{Diagramme de la vue}
    \label{view}
\end{figure}

Le tout s’interconnecte et interagit donc pour faire fonctionner l’ensemble. Mais pour alimenter l’application en données exploitables, des services ont dû être mis en place. Il en existe deux qui servent de sources de données au programme. Le premier est le service GPS, il permet à l’application de récupérer la position du dispositif Android. Dans un premier temps, ce service été basé uniquement sur les données matérielles de l’appareil mais afin d’améliorer la précision, l’utilisation des Google Services a été choisie. Les Google Services utilisent à la fois les données GPS pures mais aussi leur historique ainsi que les données du gyroscope et de l’accéléromètre pour déterminer la position avec une plus grande précision. Ceci s’est avéré indispensable pour une localisation correcte en intérieur.

Le deuxième service est celui responsable de la communication avec le serveur et qui permet de le requêter. Ainsi ce service permet à la fois d’envoyer sa propre position au serveur mais aussi de récupérer la liste des utilisateurs connectés et leurs informations. Ce service exécute en parallèle du thread principal des requêtes http sur le web service WatchDogZZ. Pour ce faire, il utilise la bibliothèque Volley qui permet d’effectuer simplement des communications sur le réseau.

Le patron de conception majeur dans ce système applicatif est celui observer-observable : il permet à une des entités d’être prévenue par une autre après inscription qu’un traitement a été effectué. Ceci permet le parallélisme des services et de ne pas bloquer l’interface graphique lors d’un traitement long (communication réseau).

L’ensemble est sécurisé par le système d’authentification Google. Ce choix a été motivé par la simplicité puisque le client étant sur un système Android, il possède forcément un compte Google associé. Il suffit alors d’effectuer une requête sur les serveurs de Google avec les informations d’authentification de l’appareil pour récupérer un token validant l’identité de l’utilisateur. Ce token est ensuite transmis lors de chaque requête au serveur WatchDogZZ en vue de vérifier l’identité du demandeur (voir la figure \ref{token}).

\begin{figure}[H]
    \centering
    \includegraphics[width=\textwidth]{./img/android-token.png}
    \caption{Utilisation du token Google}
    \label{token}
\end{figure}

La carte du maraudeur est une vue qui a été créée spécialement pour l’application. Elle se base sur une SurfaceView reposant sur de l’OpenGL ES 2. L’intérêt était à la fois de pouvoir dessiner en deux mais aussi trois dimensions. Un Renderer spécial a été implémenté ainsi que des managers de ressources 3D. Il est ainsi possible de gérer cette vue comme un observer du UserManager. La vue pourra par la suite afficher les scènes 3D avec les différents éléments à chaque notification. Les différentes classes entrant en jeu sont visibles sur la figure \ref{3d} ainsi que leur rôle dans le MVC.

\begin{figure}[H]
    \centering
    \includegraphics[width=\textwidth]{./img/android-3d.png}
    \caption{Fonctionnement de la 3D Android}
    \label{3d}
\end{figure}

La conception de l’application Android est très simple mais fait intervenir de nombreux éléments et services qui touchent énormément d’aspects de la programmation Android. Il existe sur les versions les plus récentes du Framework des fonctionnalités plus intéressantes et performantes toutefois le choix a été fait d’essayer de faire l’application la plus diffusable possible et donc de supporter un maximum d’appareil. Au début du développement, la version choisie était la 9 mais suite à des contraintes inévitables de développement nous avons dû monter à la version 12. Ceci reste correct d’autant plus que cela représente toujours plus de 99\% de parts de marché.

\subsection{Technologies utilisées}

Cette partie détaille les technologies ainsi que les outils de génie logiciel utilisés dans le développement de l’application.

    \subsubsection{Service Web}

Pour la conception du Service Web, nous avions besoin d'une technologie disposant des caractéristiques suivantes :
\begin{itemize}
    \item Facile à utiliser ;
    \item Disposant de nombreuses fonctionnalités ;
    \item Rapide à l'exécution ;
    \item Exécution légère sur serveur ;
    \item Configuration rapide.
\end{itemize}

Toutes ces caractéristiques se retrouvent avec le framework NodeJS \cite{nodejs}. C'est un framework Javascript qui dispose de nombreuses librairies installables à l'aide du gestionnaire de modules NPM \cite{npmjs}. De plus, étant donné que l'un d'entre-nous avait déjà utilisé une telle technologie, cela nous permettait de démarrer plus rapidement.

Le gestionnaire de modules NPM permet d'effectuer plusieurs choses.
Premièrement c'est lui qui va permettre l'installation des modules nécéssaires au bon fonctionnement du Service. Ensuite, il va se charger de résoudre les dépendances entre modules. C'est à dire que si un module à besoin d'un autre module pour fonctionner, alors celui-ci sera installé automatiquement.
Enfin, il est possible de disposer de plusieurs listes de modules à installer :
\begin{description}
    \item [Production] : comporte les modules nécéssaires au lancement du Service en mode production, donc sans les outils de debug ;
    \item [Dev] : comporte les modules installés en production ainsi que des modules complémentaires utilisés lors de la conception du Service ou à des fins de debuggage.
\end{description}

Afin de mettre en place notre Service Web, nous avons utilisés plusieurs modules :
\begin{description}
    \item [Body-parser] : parser le contenu JSON des requêtes ;
    \item [Express] : créer un serveur Http ou Https ;
    \item [Jasmine] : effectuer des tests de spécifications ;
    \item [Letsencrypt-express] : gérer les certificats Https du serveur ;
    \item [Mongodb] : système de gestion de base de données ;
    \item [Request] : effectuer des requêtes http ou https ;
    \item [Winston] : faire des logs sur plusieurs niveaux (info, error, warning, debug).
\end{description}

    \subsubsection{Android}

Le développement d'une application Android s'effectue généralement en utilisant le JAVA ainsi qu'Android Studio \cite{androidstudio}, permettant d'inclure les bibliothèques Android.
Ici, nous avons donc utilisé ces technologies pour permettre une intégration parfaite sur le système Android.

    \subsubsection{Intégration continue}

Lors de l'ajout de la phase d'intégration continue pour notre solution, nous avions plusieurs choix :
\begin{itemize}
    \item Travis CI ;
    \item Circle CI ;
    \item Amazon AWS CodePipeline ;
    \item Jenkins.
\end{itemize}

Les outils Jenkins ou CodePipeline sont des outils largement utilisés dans les industries et permettent une customization trés fine des opérations de test et de déploiement. Cependant, ceci nécéssite une configuration avancée des outils et parfois un serveur sur lequel placer l'outil (par exemple pour Jenkins). Du fait que nous n'avions pas besoin d'une configuration particulière pour tester et produire notre solution, nous nous sommes orienté vers l'outil \textbf{Travis CI}, qui permet de gérer aussi bien \textbf{Android} que \textbf{NodeJS}.

Cet outil s'utilie trés facilement avec la plateforme GitHub~\cite{github} que nous avons utilisé pour gérer le code source des différentes parties de la solution. Il faut simplement ajouter le service d'intégration pour chaque dépôt contenant le code à tester et déployer.

La configuration de ce service s'effectue à l'aide d'un fichier \textbf{.travis.yml} dont un exemple est donné dans le code~\ref{travisyml}.

\lstset{language=sh}
\begin{lstlisting}[caption=Exemple de fichier de configuration Travis (service web), label=travisyml]
language: node_js
node_js:
    - "node"

before_install: # Install the dev dependencies
    - sudo apt-get -qq update
    - sudo apt-get install -y mongodb-org

install:
    - npm install -d

before_script: # Run the server in background
    - npm start &
    - ./start_database.sh &

script:
    - npm test

after_script: # Stop the server launched and the database
    - kill %1
    - kill %2
\end{lstlisting}

Nous pouvons ainsi indiquer quelle version d'un framework utiliser, ajouter des paquets, effectuer des actions supplémentaires (par exemple lancer la base de données dans le cas du service web).

Le déploiement dans le cadre de l’application Android ne se fera pas sur le Google Play mais sur GitHub Releases pour des questions de budget. Un \textbf{.apk}, qui est le format des applications sous Android, sera déposé automatiquement par le serveur d’intégration sur \textbf{GitHub Releases}, et accéssible depuis la page principal du dépôt Android.

En ce qui concerne le déploiement du service web, celui-ci était auparavant associé à Amazon AWS OpsWorks, ce qui permettait de créer automatiquement une machine virtuelle Unix exécutant le service web ainsi que la base de données lors tests d'intégration réussis. En raison du dépassement de l'offre gratuite proposée par Amazon, nous avons décidé de prendre en main cette étape et l'effectuer manuellement sur un serveur personnel.

\subsection{Fonctionnalités introduites par la solution}

    \subsubsection{Fonctionnalités du Service Web}

Requetes sur le web service
Des URL sont mises à disposition par le service et permettent d'effectuer certaines tâches. Le passage de paramètres pour ces URL se fait directement dans le corps de la requête sous forme de JSON.
Les réponses du service sont sous forme d'objet JSON dans le corps de la réponse. Les réponses contiennent les champs suivants :

\lstset{language=Javascript}
\begin{lstlisting}[caption=Corps de la réponse serveur]
{
    'status': 'ok' / 'fail', // L'état de la requête
    'error': 'description' // Une description de l'erreur s'il y en a une
}
\end{lstlisting}


La plupart des requêtes décrites nécéssitent que l'utilisateur soit authentifié.
/login
La première requête à effectuer sur le service doit s'effectuer avec la méthode POST sur cette URL. Pour se connecter, l'utilisateur doit envoyer les paramètres suivants: 
\lstset{language=Javascript}
\begin{lstlisting}[caption=Corps de la requête login]
{
    'name': 'username', // Le nom de l'utilisateur à connecter
    'location': [1.0, 2.0, 3.0] // La position courrante de l'utilisateur
}
\end{lstlisting}

/who
Cette URL permet de récupérer en méthode GET une liste de noms de personnes actuellement en ligne. Le retour est sous la forme suivante:
\lstset{language=Javascript}
\begin{lstlisting}[caption=Corps de la requête who]
{
    'list': [
        'userName1',
        'userName2',
        'userName3'
    ]
}
\end{lstlisting}

/where
En utilisant la méthode GET, cette URL retourne la liste des utilisateurs ainsi que leur position. En utilisant la méthode POST et en passant les paramètres adéquats, l'utilisateur peut mettre à jour sa position.
En méthode GET, le service renvoie une liste des utilisateurs connectés avec leurs positions :
\lstset{language=Javascript}
\begin{lstlisting}[caption=Corps de la requête where GET]
{
    'list': [
        {
            'name': 'username1',
            'location': [1.0, 2.0, 3.0]
        },
        {
            'name': 'username2',
            'location': [1.0, 2.0, 3.0]
        },
        {
            'name': 'username3',
            'location': [1.0, 2.0, 3.0]
        }
    ]
}
\end{lstlisting}

En utilisant la méthode POST, l'utilisateur met à jour sa position. Les paramètres à envoyer sont les suivants :

\lstset{language=Javascript}
\begin{lstlisting}[caption=Corps de la requête where POST]
{
    'name': 'username',
    'location': [1.0, 2.0, 3.0]
}
\end{lstlisting}

Si l'utilisateur n'est pas connecté au moment de mettre à jour sa position, le serveur va tenter de l'authentifier et sauvegarder sa position.
Partie administration
Les fonctionnalités minimales pour l'application d'administration sont :
\begin{itemize}
    \item visualisation des logs du serveur
    \item visualisation du contenu de la base de donnée
\end{itemize}

Ensuite les fonctionnalités avancées pourront être :
\begin{itemize}
    \item gestion des utilisateurs
\end{itemize}

    
    \input{./part/fonc-android}

\subsection{Méthodes de développement}

La méthode adoptée pour la conception de cette solution devait pouvoir convenir à un projet hétérogène et à une équipe de taille faible, un binôme. De façon générale le projet se décomposait en trois sous-projets indépendants : la partie serveur, la partie client et la partie documentation.

La partie documentation est la seule qui a réellement fait l’objet d’un travail commun avec concertation, échange de points de vue et vérification du travail de l’autre puisqu’elle a consisté en la mise au point des spécifications et en la rédaction de ce rapport. Durant la phase d’analyse et devant l’état des lieux de tout ce qui devait être fait, il a paru équitable et logique de détaché une personne sur le projet Android et une autre sur le projet serveur. Ainsi la répartition des tâches et l’expertise sur les différents projets étaient très contrôlé.

Une fois que les besoins de l’application ont été analysés et que les spécifications ont été posées, nous avons transformé ses documents en un kanban regroupant les user stories principales. Ces user stories forment l’ensemble minimal des tâches à effectuer pour avoir une application fonctionnelle répondant aux demandes fondamentales du cahier des charges. Dès lors que cette sélection a été faite, le développement des projets primitifs du client et du serveur a commencé. Le kanban utilisé est celui proposé par GitHub, toutefois nous avons vite abandonné son utilisation. Des échanges réguliers sur les outils de travail d’équipe de GitHub, que nous détaillons plus loin, ont permis de suivre l’évolution du développement linéaire de chaque projet et de rendre compte du travail effectué. L’objectif de cette période était donc de livrer une version fonctionnelle de l’application pour la mi-janvier.

Au terme de cette première période nous avons pu livrer une application répondant aux critères minimaux et permettant de suivre des personnes dans l’ISIMA. En partant de cette base fonctionnelle nous avons commencé le développement de fonctionnalités plus poussées avec une méthode un peu différente : une méthode plus agile.

Nous avons listé tous les bugs connus, les améliorations et les fonctionnalités que nous souhaitions ajouter. Ensuite nous avons commencé à fonctionner en itérations agiles d’une durée de deux semaines. En début d’itérations nous sélectionnions un ensemble d’items qui étaient des « issues » sur GitHub et nous en faisions une milestone, voir figure \ref{milestone}. Nous travaillions ensuite sur ces issues et en fin de cycle nous faisions le point sur l’itération terminée et nous rajoutions des issues en fonction de la situation. Le développement agile a permis de rajouter des fonctionnalités avancées sur deux ou trois itérations.

\begin{figure}[H]
    \centering
    \includegraphics[width=\textwidth]{./img/issues.png}
    \caption{Liste d'issues (en haut) suivie de la milestone de l'itération 1 (en bas)}
    \label{milestone}
\end{figure}

L’ensemble de ces éléments fait que nous avons pu développer étape par étape une application qui semblait très complexe à mettre en place. Sans cette méthode nous aurions peut-être été perdu devant tout le travail mais dans les faits, malgré quelques difficultés nous avions toujours une version fonctionnelle qui répondait aux critères minimaux du cahier des charges.

Finalement, le planning final ressemble sans entrer dans les détails au planning initial mais il comporte quand même des différences assez remarquables. Tout d’abord, la phase de développement d’une base a été un peu plus longue que prévue. Ensuite la phase d’ajout de fonctionnalités avancées a pris la forme d’un développement agile mais le temps d’installer ce processus nous n’avons eu le temps de faire que deux itérations. L’ensemble est représenté sur la figure \ref{ganttfinal}.

Nous avons maintenant explorer tous les aspects de la conception de notre solution et nous allons enfin pouvoir passer à la présentation de nos résultats.

\begin{landscape}
    \begin{figure}[h]
        \centering
        \includegraphics[height=\textwidth]{../gantt_final.png}
        \caption{Diagramme de Gantt réel}
        \label{ganttfinal}
    \end{figure}
\end{landscape}
