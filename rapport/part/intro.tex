%%%%%%%%%%%%%%%%%%%%%%%%%%%%%%%%%%%%%%%%%%%%%%%%%%%%%%%%%%%%%%%%%%%%%%%%
% INTRODUCTION
%%%%%%%%%%%%%%%%%%%%%%%%%%%%%%%%%%%%%%%%%%%%%%%%%%%%%%%%%%%%%%%%%%%%%%%%

Ce rapport a été rédigé dans le cadre du projet de troisième année du cycle ingénieur à l’Institut Supérieur d’Informatique de Modélisation et de leurs Applications (ISIMA) réalisé sur une durée de 120 heures par personne. Nous avons proposé de notre propre initiative le sujet de ce projet : la conception d’une carte interactive d’un établissement. Cette idée se base sur un constat très simple : il est parfois difficile de s'orienter dans un bâtiment de grande taille et de trouver une personne en mouvement en son sein.

Ce projet s’inspire en grande partie de la carte du maraudeur de l’univers Harry Potter, carte sur laquelle il est possible de suivre en temps réel le déplacement de toutes les personnes dans l’enceinte de Poudlard, l’école des sorciers. L'objectif est donc de proposer et mettre en place une solution évolutive, innovante et pratique pour les utilisateurs afin de s'orienter dans les bâtiments de l'ISIMA. Nous pouvons penser que ce type de solution peut s'étendre à tout type de bâtiment au sein duquel il est autorisé et possible d'être localisé. Cette solution peut avoir des applications dans le domaine du secourisme, ce qui peut permettre aux sapeurs-pompiers de localiser des personnes facilement dans un bâtiment enfumé, permettre à des entreprises hébergeant des données sensibles de localiser ses visiteurs ou collaborateurs, ou encore d’optimiser les déplacements de personnes dans des bâtiments de grande taille comme une aide à l’orientation de médecins dans un hôpital ou de techniciens dans une usine.

Nous verrons que la solution mise en place s'organise en deux principaux composants. Le premier composant consistera en un service web capable de répondre à des requêtes utilisateur de type HTTP. Ces requêtes permettront aux utilisateurs d'envoyer leur position afin de la stocker sur le serveur et de l'envoyer aux autres utilisateurs qui en font la demande. Le second composant consistera en la création d'un client du service web qui affichera à la fois les données sur les lieux mais permettra aussi le suivi et l’interaction avec ses usagers. La plateforme cible choisie pour le développement du client est une plateforme mobile afin qu’il puisse être utilisé en tout lieu et à tout moment. Ces deux parties, quoi que centrales, s’articulent au sein d’un ensemble plus complet d’outils de génie logiciel donnant à la solution une identité unique et que nous détaillerons par la suite.

L'étude débutera par une partie d’études préalables plus précises sur le sujet tant au niveau du travail à fournir pour la réalisation de la solution que de l’état de l’art en la matière. Ensuite, nous détaillerons dans une seconde partie la conception de cette solution en détaillant nos méthodes et outils, pour enfin terminer ce rapport par les résultats finaux de notre travail et discuter du potentiel de notre application.
