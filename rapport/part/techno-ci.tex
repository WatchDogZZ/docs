\subsubsection{Intégration continue}

Lors de l'ajout de la phase d'intégration continue pour notre solution, nous avions plusieurs choix :
\begin{itemize}
    \item Travis CI ;
    \item Circle CI ;
    \item Amazon AWS CodePipeline ;
    \item Jenkins.
\end{itemize}

Les outils Jenkins ou CodePipeline sont des outils largement utilisés dans les industries et permettent une customization trés fine des opérations de test et de déploiement. Cependant, ceci nécéssite une configuration avancée des outils et parfois un serveur sur lequel placer l'outil (par exemple pour Jenkins). Du fait que nous n'avions pas besoin d'une configuration particulière pour tester et produire notre solution, nous nous sommes orienté vers l'outil \textbf{Travis CI}, qui permet de gérer aussi bien \textbf{Android} que \textbf{NodeJS}.

Cet outil s'utilie trés facilement avec la plateforme GitHub~\cite{github} que nous avons utilisé pour gérer le code source des différentes parties de la solution. Il faut simplement ajouter le service d'intégration pour chaque dépôt contenant le code à tester et déployer.

La configuration de ce service s'effectue à l'aide d'un fichier \textbf{.travis.yml}. Ce fichier est versionné au même titre que le code source de la solution. Ainsi, à chaque évolution de la solution, ce fichier peut également évoluer et permettre les tests et le déploiement du code concerné.
Un exemple de fichier de configuration est donné dans le code~\ref{travisyml}.

\lstset{language=sh}
\begin{lstlisting}[caption=Exemple de fichier de configuration Travis (service web), label=travisyml]
language: node_js
node_js:
    - "node"

before_install: # Install the dev dependencies
    - sudo apt-get -qq update
    - sudo apt-get install -y mongodb-org

install:
    - npm install -d

before_script: # Run the server in background
    - npm start &
    - ./start_database.sh &

script:
    - npm test

after_script: # Stop the server launched and the database
    - kill %1
    - kill %2
\end{lstlisting}

Nous pouvons ainsi indiquer quelle version d'un framework utiliser, ajouter des paquets, effectuer des actions supplémentaires (par exemple lancer la base de données dans le cas du service web).

Le déploiement dans le cadre de l’application Android ne se fera pas sur le Google Play mais sur GitHub Releases pour des questions de budget. Un \textbf{.apk}, qui est le format des applications sous Android, sera déposé automatiquement par le serveur d’intégration sur \textbf{GitHub Releases}, et accéssible depuis la page principal du dépôt Android.

En ce qui concerne le déploiement du service web, celui-ci est effectué sur une machine personnelle, exécutant \textbf{Ubuntu Server 16.10}. Cette machine dispose également d'une installation de MongoDB. Pour le déploiement (ou redéploiement) du service, un script (voir code~\ref{deploy}) se charge d'effectuer les opérations suivantes :
\begin{enumerate}
    \item Arrêt de l'ancienne instance ;
    \item Mise à jour du code du service ;
    \item Mise à jour des dépendances ;
    \item Relance du service.
\end{enumerate}

Les opérations du script sont lancées en ssh à partir d'une machine dont la clé ssh est copiée sur le serveur. Pour plus de sécurité, l'authentification par mot de passe en ssh a été désactivée. Ce mode de mise à jour du service implique une période d'indisponibilité entre les étapes 2 et 3 décrite au-dessus.

\lstset{language=sh}
\begin{lstlisting}[caption=Script de déploiement du service, label=deploy]
echo "ssh -T -i ${KEY_FILE} ${USERNAME}@${IP}"
ssh -i ${KEY_FILE} ${USERNAME}@${IP} <<'ENDSSH'
    cd server
    npm stop
    git pull
    npm install
    npm start &>/dev/null &
    exit
ENDSSH
\end{lstlisting}