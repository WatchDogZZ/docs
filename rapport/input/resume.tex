%%%%%%%%%%%%%%%%%%%%%%%%%%%%%%%%%%%%%%%%%%%%%%%%%%%%%%%%%%%%%%%%%%%%%%%%
% RÉSUMÉ
%%%%%%%%%%%%%%%%%%%%%%%%%%%%%%%%%%%%%%%%%%%%%%%%%%%%%%%%%%%%%%%%%%%%%%%%

%FR

Le travail présenté dans ce rapport concerne l'élaboration d'une solution visant à faciliter l’\textbf{orientation} des usagers au sein d'un bâtiment. Pour ceci, les utilisateurs disposent d'une \textbf{carte interactive} sur \textbf{plateforme mobile} se mettant à jour par des requêtes sur un \textbf{service web}, affichant temps réel la position des autres usagers ou de points d’intérêt. Ce projet s’inspire de la carte du \textbf{maraudeur} de l’univers Harry Potter et se découpe donc en deux parties distinctes : une partie service web, ainsi qu'une partie application mobile \textbf{Android}.

La partie service web est réalisée en utilisant le module \textbf{Express} ajouté au framework de base \textbf{NodeJS}, permettant de réaliser simplement un serveur web. D'autres modules complémentaires viennent s'ajouter pour disposer de plus de fonctionnalités. Ce service permet de traiter des requêtes soumises par les clients mobiles, ainsi que de stocker des données relatives au bon fonctionnement de l'application.

La seconde partie concernant l'application Android permet à un utilisateur d’afficher la carte interactive et se de se connecter au service web. Cette application effectue des requêtes de mise à jour de la position de l'utilisateur sur le service web. Ainsi l’application des autres utilisateurs est capable de récupérer ces \textbf{données ouvertes} et de les placer au sein sa propre carte interactive.

Les tests de cette solution ont pu montrer qu'il est possible d'afficher la position des utilisateurs sur une carte modélisée de l'ISIMA. Cependant, des erreurs de positionnement se retrouvent dans la position des utilisateurs, du fait de la mauvaise réception GPS par le mobile. Cette solution mise en place pourra faire l'objet d'améliorations futures telles que l'ajout d'itinéraires entre deux utilisateurs ou vers un point d'intérêt. L'aspect modulaire de cette solution offre de multiples possibilités d'améliorations dans le futur.

%%
Orientation, carte interactive, plateforme mobile, service web, marauder, Android, Express, NodeJS, données ouvertes.


%EN

The work introduced in this report is about a solution to make it easier for people in buildings to \textbf{orient} themselves. For that, users will be able to consult an \textbf{interactive map} displayed on a \textbf{mobile plarform}. This map is constantly updating itself by making requests on a \textbf{web service}, allowing to view in real time other users' positions or interest points. This project is inspired by Harry Potter's \textbf{marauder}'s map and is splitted into two parts : a web service part, and an \textbf{Android} mobile application.

The web service part is realized using the \textbf{Express} module added in the \textbf{NodeJS} framework, allowing to easily create a web server. Other additional modules will also be added to get more features. This service allows to process requests send by mobile clients as log as storing data relative for proper application running and updating.

The second part concerns the Android application and allows for a user to visualize the interactive map by connecting on the web service. This app makes requests to update the user's position on the web service. Thus, other users are able to recover these \textbf{open data} and place them on their own map modeling the building.

The tests on the solution demonstrated that it is possible to print the users' positions on a modeled map of ISIMA. However positioning errors are still present when locating users, due to bad GPS reception in the building. The build solution could be enhanced by adding a feature to visualize route between two user or to an interest point. The modular aspect of this solution offers several enhancement opportunities in the future.

%%
Orient, interactive map, mobile platform, web service, marauder, Android, Express, NodeJS, open data.



