\subsubsection{Service Web}

Pour la conception du Service Web, nous avions besoin d'une technologie disposant des caractéristiques suivantes :
\begin{itemize}
    \item Facile à utiliser ;
    \item Disposant de nombreuses fonctionnalités ;
    \item Rapide à l'exécution ;
    \item Exécution légère sur serveur ;
    \item Configuration rapide.
\end{itemize}

Toutes ces caractéristiques se retrouvent avec le framework NodeJS \cite{nodejs}. C'est un framework Javascript qui dispose de nombreuses librairies installables à l'aide du gestionnaire de modules NPM \cite{npmjs}. De plus, étant donné que l'un d'entre-nous avait déjà utilisé une telle technologie, cela nous permettait de démarrer plus rapidement.

Le gestionnaire de modules NPM permet d'effectuer plusieurs choses.
Premièrement c'est lui qui va permettre l'installation des modules nécéssaires au bon fonctionnement du Service. Ensuite, il va se charger de résoudre les dépendances entre modules. C'est à dire que si un module à besoin d'un autre module pour fonctionner, alors celui-ci sera installé automatiquement.
Enfin, il est possible de disposer de plusieurs listes de modules à installer :
\begin{description}
    \item [Production] : comporte les modules nécéssaires au lancement du Service en mode production, donc sans les outils de debug ;
    \item [Dev] : comporte les modules installés en production ainsi que des modules complémentaires utilisés lors de la conception du Service ou à des fins de debuggage.
\end{description}

Afin de mettre en place notre Service Web, nous avons utilisés plusieurs modules :
\begin{description}
    \item [Body-parser] : parser le contenu JSON des requêtes ;
    \item [Express] : créer un serveur Http ou Https ;
    \item [Jasmine] : effectuer des tests de spécifications ;
    \item [Letsencrypt-express] : gérer les certificats Https du serveur ;
    \item [Mongodb] : système de gestion de base de données ;
    \item [Request] : effectuer des requêtes http ou https ;
    \item [Winston] : faire des logs sur plusieurs niveaux (info, error, warning, debug).
\end{description}